\chapter{Secure coding}\label{ch:secure}

The application has been compiled with all compiler warnings enabled by passing
the \code{-Wall -Wextra} options to the compiler. Also, the \code{-Wpedantic}
option has been used to ensure the code is \emph{ISO C} compliant. No warnings
are reported by the compiler. Compiler warnings can be enabled when configuring
the source by passing the \code{-{}-enable-warnings} option to the
\code{configure} script, as explained in~\appendixref{appendix:howtocompile}.

In various parts of the code, we used the \code{assert} macro to check for the
correctness of the execution. This is manly used to check that functions are
called with all the required parameters and that some functions return the
expected values. Assertions can be enabled when configuring the source by
passing the \code{-{}-enable-assertions} to the \code{configure} script, as
explained in \appendixref{appendix:howtocompile}.

Malformed messages are handled through the \code{magic} header field. Also
checks over other header fields are performed \exgratia{maximum payload size,
valid counter, \etc}. Every string in the messages are of fixed length (except
for PEM serialized data, for which a \code{len} field is added to the message):
if not null-terminated, the terminator will be added by the receiver.

Also malformed inputs in the client application are handler by the \code{cin}
module.

\section{Memory management}\label{sec:memory}

\ldots

\subsection{Heap allocation}\label{subsec:alloc}

\ldots

\subsection{Secure memory erasing}\label{subsec:erasing}

\ldots

