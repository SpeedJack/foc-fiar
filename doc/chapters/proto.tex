\chapter{Protocol design}\label{ch:proto}

The implementation of the protocol can be found in the \code{common/protocol.c}
file.

Through this module, the application (either the client or the server) can send
and receive three different type of messages:
\begin{enumerate}
	\item \standout{Plain message}: a plaintext message. These messages are
		used only by the server to send its public certificate to the
		client;
	\item \standout{Signed message}: a plaintext message with a signature
		appended to it. The signature is generated using the \openssl{}
		EVP API with SHA-256 as hashing algorithm and RSA as signing
		algorithm. These messages are used before the GCM initialization
		during the key establishment process described in
		\secref{sec:keyxchng};
	\item \standout{GCM encrypted message}: a message encrypted and
		authenticated with GCM AES-128. These messages are used when the
		GCM module has been initialized after a session key has been
		established.
\end{enumerate}

All the messages exchanged by the application, regardless of their type, are
prefixed with the header, of a total size of 49 bytes, shown
in~\lstref{lst:msgheader}.

\lstinputlisting[language=c, label={lst:msgheader}, caption={Header prefixed to
each message of the application}]{msgheader.c}

Here, we describe each field of the header and their use:
\begin{itemize}
	\item[\standout{magic}] \emph{(4 bytes)} A magic number, always equal to
		\code{0xDECODE}.  It is used to ensure that the received message
		has a valid header;
	\item[\standout{counter}] \emph{(4 bytes)} An incremental message
		counter. It starts from \(1\) and it's incremented for each
		message sent. It is used to avoid replay attacks;
	\item[\standout{type}] \emph{(1 byte)} The message type, used to convert
		the payload to the correct data structure;
	\item[\standout{payload\_size}] \emph{(4 bytes)} Total payload size. For
		the usage of this field, see below;
	\item[\standout{nonce}] \emph{(4 bytes)} A random 4 bytes integer. It is
		use manly to add randomness to the message (why this is useful
		is explained in \secref{sec:keyxchng}. It is also used (among
		other variables) to derive a new initialization vector for GCM
		after each message exchanged (see \secref{sec:protosecurity} for
		details);
	\item[\standout{prev\_hash}] \emph{(32 bytes)}SHA-256 hash of the last
		received message.  This is used for the security of signed
		messages. See \secref{sec:protosecurity} for details.
\end{itemize}

Before GCM initialization the receiver can extract the total length of any
message by just reading the message header (of fixed size) from the socket and
then read the entire payload of size equal to \code{header.payload\_size}.

In case of signed messages, after the message (with header and payload), the
sender appends a 4 byte integer, which represents the size of the signature, and
the signature.

In case of GCM encrypted messages, the GCM tag (with a fixed size of 16 bytes)
is appended after the message.

After GCM initialization, since the messages are encrypted, there is no way for
the receiver to extract the header without knowing \emph{a priori} the size of
the entire message.

In this case, if the total message size (header plus payload) is larger than 128
bytes, the message is split into two smaller messages: the first with a total
size of 128 bytes; the second with a variable size (the header plus the
remaining payload). An header is prefixed to both messages.

If the total message size is lower than 128 bytes, the payload is padded with
zeroes until a size of 128 total bytes is reached. The padding portion of the
payload can be separated by the useful data by evaluating the
\code{payload\_size} field of the header.

This ensures that, in case of GCM message, the receiver always gets a message of
128 bytes. Then, it can determine the size of the eventual second GCM message by
reading the \code{payload\_size} field of the header.

\section{Security of the protocol}\label{sec:protosecurity}

Signed and GCM encrypted messages require some attentions in order to avoid
security flaws.

GCM needs to be initialized with a key and an initialization vector (IV).
Moreover, to be secure, the same IV must not be used more than once with the
same key.

The application we developed, derives the GCM key and the first IV using the
secret shared with the algorithm described in \secref{sec:keyxchng}. The GCM key
remains the same for the entire session, while the IV changes after every
encryption/decryption of a message.

The GCM key and the first IV are derived from the secret in this way: first, the
shared secret is hashed using SHA-256; Then, the first 16 bytes of the hash are
used as the GCM key while the last 12 bytes are used as the first IV (4 bytes
are discarded).

Then, after each message (either sent or received), a new IV is computed by
doing the SHA-256 hash of a 20 bytes array constructed by concatenating:
\begin{enumerate*}[label=\textnormal{\arabic*)}]
	\item an incremental counter;
	\item the previous IV\@;
	\item the \code{nonce} field of the header of the last message received
		or sent.
\end{enumerate*}
From the resulting hash the 12 bytes in the center (leaving 10 bytes to the left
and 10 bytes to the right) are taken as the new IV\@.

In this way the next IV can be deterministically derived from the previous IV\@.

Regarding signed message, we need to adopt some additional protection against
replay attacks. An attacker may record the messages exchanged between two
parties \exgratia{the client and the server} and resend some of the recorded
messages to one party in order to impersonate the other. In this case the
receiver will consider the message valid, since it comes with a valid signature.

As said before, the \code{nonce} field in the header ensures randomness in every
message, but this is not sufficient: the nonce is not checked against a list of
previously received nonces, so a replay attack will not be recognized (the
\code{counter} field is used to avoid replay attacks only in a single session:
it cannot prevent replays between two different sessions since when the
application is restarted the counter is initialized back to \(1\)).

To solve this issue, we have introduced the \code{prev\_hash} header field. This
field must always contains the hash of the last received message. The sender of
a message is sure about the freshness of the message he sent thanks to the
\code{nonce} field. So, when he receives the next message, he is sure about the
freshness of the \code{prev\_hash} field. This guarantees that the signed
message is fresh.

A formal description of the protocol, using BAN logic, is provided in the next
section.

\section{Key exchange protocol}\label{sec:keyxchng}

\figref{fig:keyxchng} shows the sequence diagram of a session with three
parties: Alice (client), Bob (client) and the Server.

\begin{figure}[htb]
	\includegraphics[width=\textwidth]{keyxchng}
	\caption{Sequence diagram of a sample session}\label{fig:keyxchng}
\end{figure}

In the figure only the content of the body part of messages are represented: the
header for each message is always the one shown in \lstref{lst:msgheader}.

\subsection{Formal description}\label{subsec:formal}

In the following we will simplify the protocol messages to only contains the
informations needed to guarantee the security of the protocol (in the real
protocol, each field of the header is always included even when it is not
useful).

\subsubsection{Notation}

\begin{itemize}
	\item \(A\) is a client, also named ``Alice'';
	\item \(B\) is another client, also named ``Bob'';
	\item \(S\) is the server;
	\item \(\mathit{CA}\) is the certification authority;
	\item \(N_P^n\) is a random nonce generated by the principal \(P\). The
		superscript \(n\) is used to distinguish between different
		nonces;
	\item \(\hash(X)\) means ``the hash of \(X\)'';
	\item \(\hash(\ldots)\) means ``the hash of the current message''. So,
		when we write \(\sign{\hash(\ldots)}{K}\) we mean that the
		message is hashed and then the hash is signed with \(K^{-1}\).
		This represents the work done by the EVP API of \openssl;
	\item \(\mathit{DHKEY}_P^{PQ}\) is the Diffie-Hellman public key
		generated by the principal \(P\) for use in the communication
		between \(P\) and \(Q\);
	\item \(\mathit{DH}_{params}\) are the Diffie-Hellman pre-shared
		parameters (\(p\) and \(g\));
	\item \(K_{GCM}^{PQ}\) is the shared GCM key between \(P\) and \(Q\)
		derived from the shared secret obtained from Diffie-Hellman.
\end{itemize}

\subsubsection{Idealized protocol}

Here we assume that there are no errors and the client successfully
authenticates with the server:
\begin{itemize}
	\item[M1:] \(A \sends S \qquad \sign{N_A^1, A, port}{K_A}\)
	\item[M2:] \(S \sends A \qquad N_S^1, \sign{K_S^{-1}}{K_{CA}}, \hash(M1)\)
	\item[M3:] \(S \sends A \qquad N_S^2, A, \hash(M1), \sign{\hash(\ldots)}{K_S}\)
	\item[M4:] \(A \sends S \qquad N_A^2, \mathit{DHKEY}_A^{AS}, \hash(M3), \sign{\hash(\ldots)}{K_A}\)
	\item[M5:] \(S \sends A \qquad N_S^3, \mathit{DHKEY}_S^{AS}, \hash(M4), \sign{\hash(\ldots)}{K_S}\)
\end{itemize}

At this point the session key for GCM is derived (\(K_{GCM}^{AS}\)). Here we will
show the continuation of the protocol when Alice asks to challenge Bob. For
simplicity, we do not show the message that the server sends to Bob with the
challenge request and the response for accepting/refusing it (like if the
challenge is automatically accepted by Bob):
\begin{itemize}
	\item[M6:] \(A \sends S \qquad \encrypt{N_A^3, \mathit{CHALL\_REQ}, B, \hash(M5)}{K_{GCM}^{AS}}\)
	\item[M7:] \(S \sends A \qquad \encrypt{N_S^4, \mathit{address}_B, K_B^{-1}, N_{DH}, \hash(M6)}{K_{GCM}^{AS}}\)
	\item[M8:] \(S \sends B \qquad \encrypt{N_S^5, \mathit{address}_A, K_A^{-1}, N_{DH}, \hash(\mathit{prev\,msg\,from\,B})}{K_{GCM}^{BS}}\)
	\item[M9:] \(A \sends B \qquad N_A^4, \mathit{DHKEY}_A^{AB}, \sign{\hash(\ldots)}{K_A}\)
	\item[M10:] \(B \sends A \qquad N_B^1, \mathit{DHKEY}_B^{AB}, \hash(M9), \sign{\hash(\ldots)}{K_B}\)
	\item[M11:] \(A \sends B \qquad \encrypt{N_A^5, \mathit{GAME\_MOVE}_A, \hash(M10)}{K_{GCM}^{AB}}\)
	\item[M12:] \(B \sends A \qquad \encrypt{N_B^2, \mathit{GAME\_MOVE}_B, \hash(M11)}{K_{GCM}^{AB}}\)
\end{itemize}

\subsubsection{Assumptions}

\begin{itemize}
	\item \(A \believes \fresh{N_A^n}\) for each \(n \in \mathbb{N}\)
	\item \(B \believes \fresh{N_B^n}\) for each \(n \in \mathbb{N}\)
	\item \(S \believes \fresh{N_S^n}\) for each \(n \in \mathbb{N}\)
	\item \(A \believes S \controls N_S^n\) for each \(n \in \mathbb{N}\)
	\item \(B \believes S \controls N_S^n\) for each \(n \in \mathbb{N}\)
	\item \(S \believes A \controls N_A^n\) for each \(n \in \mathbb{N}\)
	\item \(S \believes B \controls N_B^n\) for each \(n \in \mathbb{N}\)
	\item \(A \believes B \controls N_B^n\) for each \(n \in \mathbb{N}\)
	\item \(B \believes A \controls N_A^n\) for each \(n \in \mathbb{N}\)
	\item \(S \believes \pubkey{K_A} A\)
	\item \(S \believes \pubkey{K_B} B\)
	\item \(A \believes \pubkey{K_{CA}} \mathit{CA}\)
	\item \(B \believes \pubkey{K_{CA}} \mathit{CA}\)
	\item \(A \believes \mathit{CA} \controls \pubkey{K_S} S\)
	\item \(B \believes \mathit{CA} \controls \pubkey{K_S} S\)
	\item \(A \believes A \secret{\mathit{DH}_{params}} S\)
	\item \(S \believes S \secret{\mathit{DH}_{params}} A\)
	\item \(B \believes B \secret{\mathit{DH}_{params}} S\)
	\item \(S \believes S \secret{\mathit{DH}_{params}} B\)
	\item \(A \believes A \secret{\mathit{DH}_{params}} B\)
	\item \(B \believes B \secret{\mathit{DH}_{params}} A\)
	\item \(A \believes S \controls \mathit{DHKEY}_S^{AS}\)
	\item \(S \believes A \controls \mathit{DHKEY}_A^{AS}\)
	\item \(B \believes S \controls \mathit{DHKEY}_S^{BS}\)
	\item \(S \believes B \controls \mathit{DHKEY}_B^{BS}\)
	\item \(A \believes B \controls \mathit{DHKEY}_B^{AB}\)
	\item \(B \believes A \controls \mathit{DHKEY}_A^{AB}\)
\end{itemize}

\subsubsection{Objectives}

\begin{itemize}
	\item \(A \believes \pubkey{K_S} S\)
	\item \(B \believes \pubkey{K_S} S\)
	\item \(A \believes S \believes \fresh{S \sharedkey{K_{GCM}^{AS}} A}\)
	\item \(S \believes A \believes \fresh{A \sharedkey{K_{GCM}^{AS}} S}\)
	\item \(B \believes S \believes \fresh{S \sharedkey{K_{GCM}^{BS}} B}\)
	\item \(S \believes B \believes \fresh{B \sharedkey{K_{GCM}^{BS}} S}\)
	\item \(A \believes B \believes \fresh{B \sharedkey{K_{GCM}^{AB}} A}\)
	\item \(B \believes A \believes \fresh{A \sharedkey{K_{GCM}^{AB}} B}\)
	\item \(S \believes A \believes \mathit{CHALL\_REQ}\)
	\item \(A \believes S \believes \mathit{address}_B, K_B^{-1}, N_{DH}\)
	\item \(B \believes S \believes \mathit{address}_A, K_A^{-1}, N_{DH}\)
	\item \(A \believes \pubkey{K_B} B\)
	\item \(B \believes \pubkey{K_A} A\)
	\item \(A \believes B \believes \mathit{GAME\_MOVE}_B\)
	\item \(B \believes A \believes \mathit{GAME\_MOVE}_A\)
\end{itemize}

\subsubsection{Proof}

