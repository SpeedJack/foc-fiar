\section{Run the application}\label{appendix:howtorun}

Compiled binaries will be placed under the \code{server} and \code{client}
directory.

To start the server:
\begin{verbatim}
$ cd server && ./server
\end{verbatim}

To run the client:
\begin{verbatim}
$ cd client && ./client
\end{verbatim}

When running multiple clients on the same machine, use the \code{-l} option to
specify a different game port for each client.

The server needs to find all the users' public keys in a single folder. By
default, it tries to find the keys in a folder named \code{users} in the current
working directory. The folder path can be changed with the \code{-d} option.

The client needs to find the user's private key in a file named
\code{<username>.pem}. Alternatively, a different file can be specified with the
\code{-k} option.

The \code{server} and \code{client} folders already contains all the needed
files at their default location, so there should not be the need to pass any
option to the programs. The following pairs of usernames and (very secure)
passwords can be used for testing:
\begin{verbatim}
Alice:password
Bob:password
Carol:password
Daenerys:password
Erin:password
Frank:password
\end{verbatim}

Other command line options are available for both the client and the server. To
print the help message, use the \code{-h} option.
