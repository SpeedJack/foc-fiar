\section{Security of the protocol}\label{sec:protosecurity}

Signed and GCM encrypted messages require some attentions in order to avoid
security flaws.

GCM needs to be initialized with a key and an initialization vector (IV).
Moreover, to be secure, the same IV must not be used more than once with the
same key.

The application we developed, derives the GCM key and the first IV using the
secret shared with the algorithm described in \secref{sec:keyxchng}. The GCM key
remains the same for the entire session, while the IV changes after every
encryption/decryption of a message.

The GCM key and the first IV are derived from the secret in this way: first, the
shared secret is hashed using SHA-256; Then, the first 16 bytes of the hash are
used as the GCM key while the last 12 bytes are used as the first IV (4 bytes
are discarded).

Then, after each message (either sent or received), a new IV is computed by
doing the SHA-256 hash of a 20 bytes array constructed by concatenating:
\begin{enumerate*}[label=\textnormal{\arabic*)}]
	\item an incremental counter;
	\item the previous IV\@;
	\item the \code{nonce} field of the header of the last message received
		or sent.
\end{enumerate*}
From the resulting hash the 12 bytes in the center (leaving 10 bytes to the left
and 10 bytes to the right) are taken as the new IV\@.

In this way the next IV can be deterministically derived from the previous IV\@.

Regarding signed message, we need to adopt some additional protection against
replay attacks. An attacker may record the messages exchanged between two
parties \exgratia{the client and the server} and resend some of the recorded
messages to one party in order to impersonate the other. In this case the
receiver will consider the message valid, since it comes with a valid signature.

As said before, the \code{nonce} field in the header ensure randomness in every
message, but this is not sufficient: the nonce is not checked against a list of
previously received nonces, so a replay attack will not be recognized (the
\code{counter} field is used to avoid replay attacks only in a single session:
it cannot prevent replays between two different sessions since when the
application is restarted the counter is initialized back to \(1\)).

To solve this issue, we have introduced the \code{prev\_hash} header field. This
field must always contains the hash of the last received message. The sender of
a message is sure about the freshness of the message he sent thanks to the
\code{nonce} field. So, when he receives the next message, he is sure about the
freshness of the \code{prev\_hash} field. This guarantees that the signed
message is fresh.

A formal description of the protocol, using BAN logic, is provided in the next
section.
