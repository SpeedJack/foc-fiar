\chapter{Specifications}\label{ch:specs}

The application developed implements the \emph{Connect 4} (also called
\emph{Four-In-A-Row}) online game. It allows the users, registered through the
server, to play the game against each other over the network.

The application allows the users, already registered on the server, to log-in
using \emph{public key authentication}. Server and client authenticates each
other verifying that their public keys are valid. The public key of the server
is verified through a certification authority (CA).

After the log-in, a user can see all the other users connected to the server and
challenge one of them. If the challenged user accept the challenge, the game is
started and the two users initiate a \emph{peer-to-peer session} to exchange
messages regarding the game \idest{the moves they make in the game}.

The server knows when two users are playing and when their game ends. In this
way, other users that want to play can not challenge a user that is already in a
game.

The communication between the client and the server and during the
client-to-client peer-to-peer session are secured using the \openssl{} library.
In particular, when a client connects to the server or to another client, a
\emph{key negotiation algorithm}, based on the \emph{Diffie-Hellman key exchange
algorithm}, is executed. Each message exchanged during this negotiation is
authenticated using \emph{public key authentication}. When the two parties have
agreed on the session key, they encrypt each subsequent message using AES
symmetric encryption in \emph{Galois/Counter Mode} (GCM) with a key of 128 bits
(AES-128-GCM).

The application is developed using the C language and offer a command-line
interface (CLI) to the user. The user makes actions in the application by typing
commands when the \emph{prompt} (\code{>}) is shown. During the game, the
application shows the game board in a table where the \code{X} symbol represents
the discs inserted by the user while the \code{O} symbol represents the discs
inserted by the opponent.

For details on how to run and use the application,
see~\appendixref{appendix:howtorun}.
