\section{Memory management}\label{sec:memory}

In this section we describe how we manage the data in memory in the application.

\subsubsection{Heap allocation}

The \code{OPENSSL\_malloc} function is always used to allocate memory on the
heap. The memory is freed always through the \code{OPENSSL\_\{,clear\_\}free}
functions. A lot of function (especially in the common library's modules)
returns pointers to heap allocated memory. Some requires that the caller frees
the memory; others automatically handle the freeing of the memory through a
customized \code{free} function.

We checked that the application correctly frees all the allocated memory when it
is not used anymore using the \code{MEMDBG} module. Also, Valgrind with the
memcheck tool has been used in order to spot and remove memory leaks, memory
access errors and to ensure that any file descriptor is closed when the
application exits.

\subsubsection{Secure memory erasing}

The application has been compiled using the \code{-{}-fno-builtin-memset} option
to instruct the compiler to not optimize out the calls to \code{memset} when
used to clear sensitive data.

More often, the \code{OPENSSL\_clear\_free} function is used.

Only memory areas that contains sensitive data \exgratia{plaintext of ciphered
messages; shared secrets; private keys; GCM key and IV\@; \etc} is cleared.
